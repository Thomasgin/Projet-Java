\section{Introduction} 

% Contexte
Lorsque l'on souhaite capturer une image, on le fait à l'aide de capteurs (photodiode, CMOS/CCD). Ces capteurs sont sensibles aux bruits qui peut provenir de plusieurs source : arrondi de conversion, interférences électromagnétiques, agitations des électrons. \par
On retrouve ce bruit dans la photographie numérique, l'imagerie médical ainsi que dans la transmitions d'image.

% Problématique
Un des problème majeur du bruit sur les images est la dégradation des images médicales car elles doivent être analysées avec précisions. \par
Il est alors nécessaire d'identifier le bruit sur l'image et de l'éliminer afin de gagner en qualité sur celle-ci afin que l'on puisse observer les détails sur l'image.

% Objectifs
Dans ce projet, nous ne travaillerons que sur du bruit de type gaussien. Nous tenterons, grace à différentes méthode et l'algorithme de l'ACP, de trouver une méthode optimale afin de débruiter le plus efficacement une image.

% Organisation
Dans ce rapport, nous expliquerons la méthode mathématiques que nous allons utiliser pour débruiter l'image ainsi que l'implémentation d'une interface homme-machine (IHM). Nous analyserons ensuites les résultats obtenus selon le type de débruitage que nous appliquerons afin de déterminer une méthode optimal pour débruiter une image.