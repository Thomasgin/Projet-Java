\newcommand{\ABS}[1]{\LR{|}{#1}{|}}                         % mathmode              : absolute value
\newcommand{\BAR}[1]{\overline{#1}}                         % mathmode              : a bar above a letter
\newcommand{\BBR}[1]{\LR{\{}{#1}{\}}}                       % mathmode              : between brackets
\newcommand{\BLR}[3]{\left#1 #2 \right#3}                   % mathmode              : between left right
\newcommand{\CAL}[1]{\mathcal{#1}}                          % mathmode {amssymb}    : caligraphic letter
\newcommand{\COV}[1]{\text{Cov}\LR{(}{#1}{)}}               % mathmode {amsmath}    : covariance 
\newcommand{\DIS}{\displaystyle}                            % mathmode              : displaystyle
\newcommand{\EPS}{\varepsilon}                              % mathmode {amsmath}    : epsilon
\newcommand{\EQU}{\Leftrightarrow}                          % mathmode              : equivalent arrow
\newcommand{\ESL}[1]{\mathscr{#1}}                          % mathmode {amssymb}    : elegant script letter
\newcommand{\EXP}[1]{\exp\LR{(}{#1}{)}}                     % mathmode {IDK}        : exponential
\newcommand{\FAM}[2]{\LR{(}{#1}{)}_{#2}}                    % mathmode              : family
\newcommand{\IINT}[2]{\iint_{#1}^{#2}}                      % mathmode              : double integral
\newcommand{\IND}{\perp \!\!\! \perp}                       % mathmode              : independant symbol
\newcommand{\IDC}[1]{\mathds{1}_{#1}}                       % mathmode {dsfont}     : indicator function
\newcommand{\INT}[2]{\int_{#1}^{#2}}                        % mathmode              : integral
\newcommand{\LIM}[1]{\underset{#1}{\text{lim}}}             % mathmode              : limit
\newcommand{\MAX}[1]{\underset{#1}{\text{max}}}             % mathmode              : max
\newcommand{\MIN}[1]{\underset{#1}{\text{min}}}             % mathmode              : min
\newcommand{\PTD}{\partial}                                 % mathmode              : partial derivate symbol
\newcommand{\RML}[1]{\mathrm{#1}}                           % mathmode {amssymb}    : roman letter
\newcommand{\SET}[1]{\mathbb{#1}}                           % mathmode {amssymb}    : set
\newcommand{\SETBT}[3]{\mathbb{#1}_{#2}^{#3}}               % mathmode {amssymb}    : set + bot and top 
\newcommand{\SUM}[2]{\sum_{#1}^{#2}}                        % mathmode              : sum
\newcommand{\TEN}[2]{\underset{#2}{\xrightarrow{#1}}}       % mathmode              : tend to
\newcommand{\TXT}[1]{\;\text{#1}\;}                         % mathmode {amsmath}    : insert text in math mode
\newcommand{\VEC}[1]{\overrightarrow{#1}}                   % mathmode              : vector

% Create a system with an #1 lines and arguments around it and possible commentary on right side
\newenvironment{SYSTEM}[1]{\begin{array}{r @{\quad #1 \quad} l l}}{\end{array}}