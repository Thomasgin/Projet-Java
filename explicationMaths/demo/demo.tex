\documentclass{article}
\usepackage{amsmath}
\usepackage{amssymb}
\usepackage{geometry}
\geometry{a4paper, top=1.5cm, bottom=1.5cm, left=1.5cm, right=1.5cm}

\begin{document}

\section*{Démonstration de la décomposition dans une base orthonormale}

Soit $\mathcal{B} = \{u_1, \ldots, u_{s^2}\}$ une base orthonormale de $\mathbb{R}^{s^2}$ obtenue par ACP, et soit $V_k \in \mathbb{R}^{s^2}$ un vecteur représentant un patch vectorisé. On note $m_v$ le vecteur moyen des données.

\subsection*{Étape 1 : Centrage du vecteur}
On commence par centrer le vecteur $V_k$ en soustrayant le vecteur moyen :
\[
V_k^{\text{centré}} = V_k - m_v.
\]

\subsection*{Étape 2 : Projection sur la base orthonormale}
Comme $\mathcal{B}$ est une base orthonormale, tout vecteur $V_k^{\text{centré}}$ peut s'écrire comme une combinaison linéaire des vecteurs de $\mathcal{B}$. Les coefficients de cette combinaison sont donnés par les projections de $V_k^{\text{centré}}$ sur chaque vecteur de la base :
\[
V_k^{\text{centré}} = \sum_{i=1}^{s^2} \underbrace{\langle V_k^{\text{centré}}, u_i \rangle}_{\alpha_i^{(k)}} u_i,
\]
où $\langle \cdot, \cdot \rangle$ désigne le produit scalaire. Par orthonormalité de la base, on a :
\[
\langle u_i, u_j \rangle = \delta_{ij} \quad \text{avec} \quad \delta_{ij} = 
\begin{cases}
1 & \text{si } i = j, \\
0 & \text{sinon}.
\end{cases}
\]

\subsection*{Étape 3 : Expression des coefficients}
Les coefficients $\alpha_i^{(k)}$ sont calculés par :
\[
\alpha_i^{(k)} = \langle V_k^{\text{centré}}, u_i \rangle = u_i^\top (V_k - m_v),
\]
où $u_i^\top$ est la transposée de $u_i$.

\subsection*{Étape 4 : Reconstruction du vecteur original}
En ajoutant le vecteur moyen $m_v$ à la décomposition centrée, on obtient la reconstruction du vecteur original :
\[
V_k = m_v + V_k^{\text{centré}} = m_v + \sum_{i=1}^{s^2} \alpha_i^{(k)} u_i.
\]

\subsection*{Justification mathématique}
Cette décomposition repose sur le \textbf{théorème de projection orthogonale} dans un espace euclidien :
\begin{itemize}
    \item Toute base orthonormale permet de décomposer un vecteur en une somme de ses projections sur les axes de la base.
    \item Les coefficients $\alpha_i^{(k)}$ sont les coordonnées de $V_k^{\text{centré}}$ dans la base $\mathcal{B}$.
\end{itemize}

\subsection*{Application au seuillage}
Dans le cadre du débruitage, on applique un seuillage aux coefficients $\alpha_i^{(k)}$ pour réduire le bruit :
\[
Z_k = m_v + \sum_{i=1}^{s^2} \text{Seuillage}(\alpha_i^{(k)}) u_i.
\]
Cette opération permet de préserver les composantes principales du signal tout en atténuant les contributions du bruit.

\end{document}